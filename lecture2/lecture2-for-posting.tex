% Options for packages loaded elsewhere
\PassOptionsToPackage{unicode}{hyperref}
\PassOptionsToPackage{hyphens}{url}
%
\documentclass[
]{article}
\usepackage{amsmath,amssymb}
\usepackage{lmodern}
\usepackage{ifxetex,ifluatex}
\ifnum 0\ifxetex 1\fi\ifluatex 1\fi=0 % if pdftex
  \usepackage[T1]{fontenc}
  \usepackage[utf8]{inputenc}
  \usepackage{textcomp} % provide euro and other symbols
\else % if luatex or xetex
  \usepackage{unicode-math}
  \defaultfontfeatures{Scale=MatchLowercase}
  \defaultfontfeatures[\rmfamily]{Ligatures=TeX,Scale=1}
\fi
% Use upquote if available, for straight quotes in verbatim environments
\IfFileExists{upquote.sty}{\usepackage{upquote}}{}
\IfFileExists{microtype.sty}{% use microtype if available
  \usepackage[]{microtype}
  \UseMicrotypeSet[protrusion]{basicmath} % disable protrusion for tt fonts
}{}
\makeatletter
\@ifundefined{KOMAClassName}{% if non-KOMA class
  \IfFileExists{parskip.sty}{%
    \usepackage{parskip}
  }{% else
    \setlength{\parindent}{0pt}
    \setlength{\parskip}{6pt plus 2pt minus 1pt}}
}{% if KOMA class
  \KOMAoptions{parskip=half}}
\makeatother
\usepackage{xcolor}
\IfFileExists{xurl.sty}{\usepackage{xurl}}{} % add URL line breaks if available
\IfFileExists{bookmark.sty}{\usepackage{bookmark}}{\usepackage{hyperref}}
\hypersetup{
  pdftitle={Lecture 2: Chapter 3 by Kieran Healy},
  pdfauthor={Marc Kaufmann},
  hidelinks,
  pdfcreator={LaTeX via pandoc}}
\urlstyle{same} % disable monospaced font for URLs
\usepackage[margin=1in]{geometry}
\usepackage{color}
\usepackage{fancyvrb}
\newcommand{\VerbBar}{|}
\newcommand{\VERB}{\Verb[commandchars=\\\{\}]}
\DefineVerbatimEnvironment{Highlighting}{Verbatim}{commandchars=\\\{\}}
% Add ',fontsize=\small' for more characters per line
\usepackage{framed}
\definecolor{shadecolor}{RGB}{248,248,248}
\newenvironment{Shaded}{\begin{snugshade}}{\end{snugshade}}
\newcommand{\AlertTok}[1]{\textcolor[rgb]{0.94,0.16,0.16}{#1}}
\newcommand{\AnnotationTok}[1]{\textcolor[rgb]{0.56,0.35,0.01}{\textbf{\textit{#1}}}}
\newcommand{\AttributeTok}[1]{\textcolor[rgb]{0.77,0.63,0.00}{#1}}
\newcommand{\BaseNTok}[1]{\textcolor[rgb]{0.00,0.00,0.81}{#1}}
\newcommand{\BuiltInTok}[1]{#1}
\newcommand{\CharTok}[1]{\textcolor[rgb]{0.31,0.60,0.02}{#1}}
\newcommand{\CommentTok}[1]{\textcolor[rgb]{0.56,0.35,0.01}{\textit{#1}}}
\newcommand{\CommentVarTok}[1]{\textcolor[rgb]{0.56,0.35,0.01}{\textbf{\textit{#1}}}}
\newcommand{\ConstantTok}[1]{\textcolor[rgb]{0.00,0.00,0.00}{#1}}
\newcommand{\ControlFlowTok}[1]{\textcolor[rgb]{0.13,0.29,0.53}{\textbf{#1}}}
\newcommand{\DataTypeTok}[1]{\textcolor[rgb]{0.13,0.29,0.53}{#1}}
\newcommand{\DecValTok}[1]{\textcolor[rgb]{0.00,0.00,0.81}{#1}}
\newcommand{\DocumentationTok}[1]{\textcolor[rgb]{0.56,0.35,0.01}{\textbf{\textit{#1}}}}
\newcommand{\ErrorTok}[1]{\textcolor[rgb]{0.64,0.00,0.00}{\textbf{#1}}}
\newcommand{\ExtensionTok}[1]{#1}
\newcommand{\FloatTok}[1]{\textcolor[rgb]{0.00,0.00,0.81}{#1}}
\newcommand{\FunctionTok}[1]{\textcolor[rgb]{0.00,0.00,0.00}{#1}}
\newcommand{\ImportTok}[1]{#1}
\newcommand{\InformationTok}[1]{\textcolor[rgb]{0.56,0.35,0.01}{\textbf{\textit{#1}}}}
\newcommand{\KeywordTok}[1]{\textcolor[rgb]{0.13,0.29,0.53}{\textbf{#1}}}
\newcommand{\NormalTok}[1]{#1}
\newcommand{\OperatorTok}[1]{\textcolor[rgb]{0.81,0.36,0.00}{\textbf{#1}}}
\newcommand{\OtherTok}[1]{\textcolor[rgb]{0.56,0.35,0.01}{#1}}
\newcommand{\PreprocessorTok}[1]{\textcolor[rgb]{0.56,0.35,0.01}{\textit{#1}}}
\newcommand{\RegionMarkerTok}[1]{#1}
\newcommand{\SpecialCharTok}[1]{\textcolor[rgb]{0.00,0.00,0.00}{#1}}
\newcommand{\SpecialStringTok}[1]{\textcolor[rgb]{0.31,0.60,0.02}{#1}}
\newcommand{\StringTok}[1]{\textcolor[rgb]{0.31,0.60,0.02}{#1}}
\newcommand{\VariableTok}[1]{\textcolor[rgb]{0.00,0.00,0.00}{#1}}
\newcommand{\VerbatimStringTok}[1]{\textcolor[rgb]{0.31,0.60,0.02}{#1}}
\newcommand{\WarningTok}[1]{\textcolor[rgb]{0.56,0.35,0.01}{\textbf{\textit{#1}}}}
\usepackage{graphicx}
\makeatletter
\def\maxwidth{\ifdim\Gin@nat@width>\linewidth\linewidth\else\Gin@nat@width\fi}
\def\maxheight{\ifdim\Gin@nat@height>\textheight\textheight\else\Gin@nat@height\fi}
\makeatother
% Scale images if necessary, so that they will not overflow the page
% margins by default, and it is still possible to overwrite the defaults
% using explicit options in \includegraphics[width, height, ...]{}
\setkeys{Gin}{width=\maxwidth,height=\maxheight,keepaspectratio}
% Set default figure placement to htbp
\makeatletter
\def\fps@figure{htbp}
\makeatother
\setlength{\emergencystretch}{3em} % prevent overfull lines
\providecommand{\tightlist}{%
  \setlength{\itemsep}{0pt}\setlength{\parskip}{0pt}}
\setcounter{secnumdepth}{-\maxdimen} % remove section numbering
\ifluatex
  \usepackage{selnolig}  % disable illegal ligatures
\fi

\title{Lecture 2: Chapter 3 by Kieran Healy}
\author{Marc Kaufmann}
\date{Feburary 2022}

\begin{document}
\maketitle

\hypertarget{note-to-the-interested-student}{%
\subsection{Note to the interested
student}\label{note-to-the-interested-student}}

Try to follow along by typing it yourself, adding comments as you make
mistakes or realize things. Write the code out in chunks:

\hypertarget{how-ggplot-works}{%
\subsection{How Ggplot Works}\label{how-ggplot-works}}

\begin{Shaded}
\begin{Highlighting}[]
\FunctionTok{library}\NormalTok{(tidyverse)}
\end{Highlighting}
\end{Shaded}

\begin{verbatim}
## -- Attaching packages --------------------------------------- tidyverse 1.3.1 --
\end{verbatim}

\begin{verbatim}
## v ggplot2 3.3.5     v purrr   0.3.4
## v tibble  3.1.6     v dplyr   1.0.8
## v tidyr   1.2.0     v stringr 1.4.0
## v readr   2.1.2     v forcats 0.5.1
\end{verbatim}

\begin{verbatim}
## -- Conflicts ------------------------------------------ tidyverse_conflicts() --
## x dplyr::filter() masks stats::filter()
## x dplyr::lag()    masks stats::lag()
\end{verbatim}

\begin{Shaded}
\begin{Highlighting}[]
\CommentTok{\#install.packages("tidyverse")}
\end{Highlighting}
\end{Shaded}

The code specifies the connections between the variables in the data on
one hand and the colors, points, and shapes you see on the screen. These
logical connections are called \emph{aesthetic mappings} or simply
\emph{aesthetics}.

How to use ggplot:

\begin{itemize}
\tightlist
\item
  \texttt{data\ =\ gapminder}: Tell it what your data is
\item
  \texttt{mapping\ =\ aes(...)}: How to map the variables in the data to
  aesthetics

  \begin{itemize}
  \tightlist
  \item
    axes, size of points, intensities of colors, which colors, shape of
    points, lines/points
  \end{itemize}
\item
  Then say what type of plot you want:

  \begin{itemize}
  \tightlist
  \item
    boxplot, scatterplot, histogram, \ldots{}
  \item
    these are called `geoms' in ggplot's grammar, such as
    \texttt{geom\_point()} giving scatter plots
  \end{itemize}
\end{itemize}

\begin{Shaded}
\begin{Highlighting}[]
\FunctionTok{library}\NormalTok{(ggplot2)}
\NormalTok{... }\SpecialCharTok{+} \FunctionTok{geom\_point}\NormalTok{() }\CommentTok{\# Produces scatterplots}
\NormalTok{... }\SpecialCharTok{+} \FunctionTok{geom\_bar}\NormalTok{() }\CommentTok{\# Bar plots}
\NormalTok{... }\SpecialCharTok{+} \FunctionTok{geom\_boxplot}\NormalTok{() }\CommentTok{\# boxplots}
\NormalTok{... }\CommentTok{\# }
\end{Highlighting}
\end{Shaded}

You link these steps by \emph{literally} adding them together with
\texttt{+} as we'll see.

\textbf{Exercise:} What other types of plots are there? Try to find
several more \texttt{geom\_} functions.

\hypertarget{mappings-link-data-to-things-you-see}{%
\subsection{Mappings Link Data to Things You
See}\label{mappings-link-data-to-things-you-see}}

\begin{Shaded}
\begin{Highlighting}[]
\FunctionTok{library}\NormalTok{(gapminder)}
\FunctionTok{library}\NormalTok{(ggplot2)}
\NormalTok{gapminder}
\end{Highlighting}
\end{Shaded}

\begin{verbatim}
## # A tibble: 1,704 x 6
##    country     continent  year lifeExp      pop gdpPercap
##    <fct>       <fct>     <int>   <dbl>    <int>     <dbl>
##  1 Afghanistan Asia       1952    28.8  8425333      779.
##  2 Afghanistan Asia       1957    30.3  9240934      821.
##  3 Afghanistan Asia       1962    32.0 10267083      853.
##  4 Afghanistan Asia       1967    34.0 11537966      836.
##  5 Afghanistan Asia       1972    36.1 13079460      740.
##  6 Afghanistan Asia       1977    38.4 14880372      786.
##  7 Afghanistan Asia       1982    39.9 12881816      978.
##  8 Afghanistan Asia       1987    40.8 13867957      852.
##  9 Afghanistan Asia       1992    41.7 16317921      649.
## 10 Afghanistan Asia       1997    41.8 22227415      635.
## # ... with 1,694 more rows
\end{verbatim}

\begin{Shaded}
\begin{Highlighting}[]
\NormalTok{p }\OtherTok{\textless{}{-}} \FunctionTok{ggplot}\NormalTok{(}\AttributeTok{data =}\NormalTok{ gapminder,}
            \AttributeTok{mapping =} \FunctionTok{aes}\NormalTok{(}\AttributeTok{x =}\NormalTok{ gdpPercap, }\AttributeTok{y =}\NormalTok{ lifeExp))}
\NormalTok{p }\SpecialCharTok{+} \FunctionTok{geom\_point}\NormalTok{()}
\end{Highlighting}
\end{Shaded}

\includegraphics{lecture2-for-posting_files/figure-latex/unnamed-chunk-2-1.pdf}

In detail:

\begin{itemize}
\tightlist
\item
  \texttt{data\ =\ gapminder} tells ggplot to use gapminder dataset, so
  if variable names are mentioned, they should be looked up in gapminder
\item
  \texttt{mapping\ =\ aes(...)} shows that the mapping is a function
  call. Simply accept that this is how you write it

  \begin{itemize}
  \tightlist
  \item
    Kieran Healy: ``The \texttt{mapping\ =\ aes(...)} argument
    \emph{links variables} to \emph{things you will see} on the plot''
  \end{itemize}
\item
  \texttt{aes(x\ =\ gdpPercap,\ y\ =\ lifeExp)} maps the GDP data onto
  \texttt{x}, which is a known aesthetic (the x-coordinate) and life
  expectancy data onto \texttt{x}

  \begin{itemize}
  \tightlist
  \item
    \texttt{x} and \texttt{y} are predefined names that are used by
    \texttt{ggplot} and friends
  \end{itemize}
\end{itemize}

Importantly, mappings don't say \emph{what} color or shape some variable
will have. Rather, it says that a given column/variable will be mapped
\emph{to} color or \emph{to} shape: thus \texttt{color\ =\ gender} means
that different genders will be displayed by different colors.

\begin{Shaded}
\begin{Highlighting}[]
\FunctionTok{str}\NormalTok{(p)}
\end{Highlighting}
\end{Shaded}

\begin{verbatim}
## List of 9
##  $ data       : tibble [1,704 x 6] (S3: tbl_df/tbl/data.frame)
##   ..$ country  : Factor w/ 142 levels "Afghanistan",..: 1 1 1 1 1 1 1 1 1 1 ...
##   ..$ continent: Factor w/ 5 levels "Africa","Americas",..: 3 3 3 3 3 3 3 3 3 3 ...
##   ..$ year     : int [1:1704] 1952 1957 1962 1967 1972 1977 1982 1987 1992 1997 ...
##   ..$ lifeExp  : num [1:1704] 28.8 30.3 32 34 36.1 ...
##   ..$ pop      : int [1:1704] 8425333 9240934 10267083 11537966 13079460 14880372 12881816 13867957 16317921 22227415 ...
##   ..$ gdpPercap: num [1:1704] 779 821 853 836 740 ...
##  $ layers     : list()
##  $ scales     :Classes 'ScalesList', 'ggproto', 'gg' <ggproto object: Class ScalesList, gg>
##     add: function
##     clone: function
##     find: function
##     get_scales: function
##     has_scale: function
##     input: function
##     n: function
##     non_position_scales: function
##     scales: NULL
##     super:  <ggproto object: Class ScalesList, gg> 
##  $ mapping    :List of 2
##   ..$ x: language ~gdpPercap
##   .. ..- attr(*, ".Environment")=<environment: R_GlobalEnv> 
##   ..$ y: language ~lifeExp
##   .. ..- attr(*, ".Environment")=<environment: R_GlobalEnv> 
##   ..- attr(*, "class")= chr "uneval"
##  $ theme      : list()
##  $ coordinates:Classes 'CoordCartesian', 'Coord', 'ggproto', 'gg' <ggproto object: Class CoordCartesian, Coord, gg>
##     aspect: function
##     backtransform_range: function
##     clip: on
##     default: TRUE
##     distance: function
##     expand: TRUE
##     is_free: function
##     is_linear: function
##     labels: function
##     limits: list
##     modify_scales: function
##     range: function
##     render_axis_h: function
##     render_axis_v: function
##     render_bg: function
##     render_fg: function
##     setup_data: function
##     setup_layout: function
##     setup_panel_guides: function
##     setup_panel_params: function
##     setup_params: function
##     train_panel_guides: function
##     transform: function
##     super:  <ggproto object: Class CoordCartesian, Coord, gg> 
##  $ facet      :Classes 'FacetNull', 'Facet', 'ggproto', 'gg' <ggproto object: Class FacetNull, Facet, gg>
##     compute_layout: function
##     draw_back: function
##     draw_front: function
##     draw_labels: function
##     draw_panels: function
##     finish_data: function
##     init_scales: function
##     map_data: function
##     params: list
##     setup_data: function
##     setup_params: function
##     shrink: TRUE
##     train_scales: function
##     vars: function
##     super:  <ggproto object: Class FacetNull, Facet, gg> 
##  $ plot_env   :<environment: R_GlobalEnv> 
##  $ labels     :List of 2
##   ..$ x: chr "gdpPercap"
##   ..$ y: chr "lifeExp"
##  - attr(*, "class")= chr [1:2] "gg" "ggplot"
\end{verbatim}

\begin{Shaded}
\begin{Highlighting}[]
\FunctionTok{str}\NormalTok{(p }\SpecialCharTok{+} \FunctionTok{geom\_point}\NormalTok{())}
\end{Highlighting}
\end{Shaded}

\begin{verbatim}
## List of 9
##  $ data       : tibble [1,704 x 6] (S3: tbl_df/tbl/data.frame)
##   ..$ country  : Factor w/ 142 levels "Afghanistan",..: 1 1 1 1 1 1 1 1 1 1 ...
##   ..$ continent: Factor w/ 5 levels "Africa","Americas",..: 3 3 3 3 3 3 3 3 3 3 ...
##   ..$ year     : int [1:1704] 1952 1957 1962 1967 1972 1977 1982 1987 1992 1997 ...
##   ..$ lifeExp  : num [1:1704] 28.8 30.3 32 34 36.1 ...
##   ..$ pop      : int [1:1704] 8425333 9240934 10267083 11537966 13079460 14880372 12881816 13867957 16317921 22227415 ...
##   ..$ gdpPercap: num [1:1704] 779 821 853 836 740 ...
##  $ layers     :List of 1
##   ..$ :Classes 'LayerInstance', 'Layer', 'ggproto', 'gg' <ggproto object: Class LayerInstance, Layer, gg>
##     aes_params: list
##     compute_aesthetics: function
##     compute_geom_1: function
##     compute_geom_2: function
##     compute_position: function
##     compute_statistic: function
##     computed_geom_params: NULL
##     computed_mapping: NULL
##     computed_stat_params: NULL
##     data: waiver
##     draw_geom: function
##     finish_statistics: function
##     geom: <ggproto object: Class GeomPoint, Geom, gg>
##         aesthetics: function
##         default_aes: uneval
##         draw_group: function
##         draw_key: function
##         draw_layer: function
##         draw_panel: function
##         extra_params: na.rm
##         handle_na: function
##         non_missing_aes: size shape colour
##         optional_aes: 
##         parameters: function
##         required_aes: x y
##         setup_data: function
##         setup_params: function
##         use_defaults: function
##         super:  <ggproto object: Class Geom, gg>
##     geom_params: list
##     inherit.aes: TRUE
##     layer_data: function
##     map_statistic: function
##     mapping: NULL
##     position: <ggproto object: Class PositionIdentity, Position, gg>
##         compute_layer: function
##         compute_panel: function
##         required_aes: 
##         setup_data: function
##         setup_params: function
##         super:  <ggproto object: Class Position, gg>
##     print: function
##     setup_layer: function
##     show.legend: NA
##     stat: <ggproto object: Class StatIdentity, Stat, gg>
##         aesthetics: function
##         compute_group: function
##         compute_layer: function
##         compute_panel: function
##         default_aes: uneval
##         extra_params: na.rm
##         finish_layer: function
##         non_missing_aes: 
##         optional_aes: 
##         parameters: function
##         required_aes: 
##         retransform: TRUE
##         setup_data: function
##         setup_params: function
##         super:  <ggproto object: Class Stat, gg>
##     stat_params: list
##     super:  <ggproto object: Class Layer, gg> 
##  $ scales     :Classes 'ScalesList', 'ggproto', 'gg' <ggproto object: Class ScalesList, gg>
##     add: function
##     clone: function
##     find: function
##     get_scales: function
##     has_scale: function
##     input: function
##     n: function
##     non_position_scales: function
##     scales: list
##     super:  <ggproto object: Class ScalesList, gg> 
##  $ mapping    :List of 2
##   ..$ x: language ~gdpPercap
##   .. ..- attr(*, ".Environment")=<environment: R_GlobalEnv> 
##   ..$ y: language ~lifeExp
##   .. ..- attr(*, ".Environment")=<environment: R_GlobalEnv> 
##   ..- attr(*, "class")= chr "uneval"
##  $ theme      : list()
##  $ coordinates:Classes 'CoordCartesian', 'Coord', 'ggproto', 'gg' <ggproto object: Class CoordCartesian, Coord, gg>
##     aspect: function
##     backtransform_range: function
##     clip: on
##     default: TRUE
##     distance: function
##     expand: TRUE
##     is_free: function
##     is_linear: function
##     labels: function
##     limits: list
##     modify_scales: function
##     range: function
##     render_axis_h: function
##     render_axis_v: function
##     render_bg: function
##     render_fg: function
##     setup_data: function
##     setup_layout: function
##     setup_panel_guides: function
##     setup_panel_params: function
##     setup_params: function
##     train_panel_guides: function
##     transform: function
##     super:  <ggproto object: Class CoordCartesian, Coord, gg> 
##  $ facet      :Classes 'FacetNull', 'Facet', 'ggproto', 'gg' <ggproto object: Class FacetNull, Facet, gg>
##     compute_layout: function
##     draw_back: function
##     draw_front: function
##     draw_labels: function
##     draw_panels: function
##     finish_data: function
##     init_scales: function
##     map_data: function
##     params: list
##     setup_data: function
##     setup_params: function
##     shrink: TRUE
##     train_scales: function
##     vars: function
##     super:  <ggproto object: Class FacetNull, Facet, gg> 
##  $ plot_env   :<environment: R_GlobalEnv> 
##  $ labels     :List of 2
##   ..$ x: chr "gdpPercap"
##   ..$ y: chr "lifeExp"
##  - attr(*, "class")= chr [1:2] "gg" "ggplot"
\end{verbatim}

\textbf{Exercise:} Make sure that your knitted version doesn't include
all the output from the \texttt{str(...)} commands, it's too tedious.
(Hint: click on the `gear' symbol in the code block above and then on
the link to `Chunk Options' to get help with chunk options.)

Finally, we add a \emph{layer}. This says how some data gets turned into
concrete visual aspects.

\begin{Shaded}
\begin{Highlighting}[]
\NormalTok{p }\SpecialCharTok{+} \FunctionTok{geom\_point}\NormalTok{()}
\end{Highlighting}
\end{Shaded}

\includegraphics{lecture2-for-posting_files/figure-latex/scatter_plot-1.pdf}

\begin{Shaded}
\begin{Highlighting}[]
\NormalTok{p }\SpecialCharTok{+} \FunctionTok{geom\_smooth}\NormalTok{()}
\end{Highlighting}
\end{Shaded}

\begin{verbatim}
## `geom_smooth()` using method = 'gam' and formula 'y ~ s(x, bs = "cs")'
\end{verbatim}

\includegraphics{lecture2-for-posting_files/figure-latex/scatter_plot-2.pdf}

\textbf{Note:} Both geom's use the same mapping, where the x-axis
represents \ldots{} and the y-axis \ldots{} based on the mappings you
defined before. But the first one maps the data to individual points,
the other one maps it to a smooth line with error ranges.

We get a message that tells us that \texttt{geom\_smooth()} is using the
method = `gam', so presumably we can use other methods. Let's see if we
can figure out which other methods there are.

\begin{Shaded}
\begin{Highlighting}[]
\NormalTok{?geom\_smooth}
\NormalTok{p }\SpecialCharTok{+} \FunctionTok{geom\_point}\NormalTok{() }\SpecialCharTok{+} \FunctionTok{geom\_smooth}\NormalTok{() }\SpecialCharTok{+} \FunctionTok{geom\_smooth}\NormalTok{(}\AttributeTok{method =}\NormalTok{ ...) }\SpecialCharTok{+} \FunctionTok{geom\_smooth}\NormalTok{(}\AttributeTok{method =}\NormalTok{ ...)}
\NormalTok{p }\SpecialCharTok{+} \FunctionTok{geom\_point}\NormalTok{() }\SpecialCharTok{+} \FunctionTok{geom\_smooth}\NormalTok{() }\SpecialCharTok{+} \FunctionTok{geom\_smooth}\NormalTok{(}\AttributeTok{method =}\NormalTok{ ...) }\SpecialCharTok{+} \FunctionTok{geom\_smooth}\NormalTok{(}\AttributeTok{method =}\NormalTok{ ..., }\AttributeTok{color =} \StringTok{"red"}\NormalTok{)}
\end{Highlighting}
\end{Shaded}

You may start to see why ggplots way of breaking up tasks is quite
powerful: the geometric objects (long for geoms) can all reuse the
\emph{same} mapping of data to aesthetics, yet the results are quite
different. And if we want later geoms to use different mappings, then we
can override them -- but it isn't necessary.

One thing about the data is that most of it is bunched to the left. If
we instead used a logarithmic scale, we should be able to spread the
data out better.

\begin{Shaded}
\begin{Highlighting}[]
\NormalTok{p }\SpecialCharTok{+} \FunctionTok{geom\_point}\NormalTok{() }\SpecialCharTok{+} \FunctionTok{geom\_smooth}\NormalTok{(}\AttributeTok{method =} \StringTok{"lm"}\NormalTok{) }\SpecialCharTok{+} \FunctionTok{scale\_x\_log10}\NormalTok{()}
\end{Highlighting}
\end{Shaded}

\begin{verbatim}
## `geom_smooth()` using formula 'y ~ x'
\end{verbatim}

\includegraphics{lecture2-for-posting_files/figure-latex/scale_coordinates-1.pdf}

\textbf{Exercise:} Describe what the \texttt{scale\_x\_log10()} does.
Why is it a more evenly distributed cloud of points now? (2-3
sentences.)

Nice! The x-axis now has scientific notation, let's change that.

\begin{Shaded}
\begin{Highlighting}[]
\FunctionTok{library}\NormalTok{(scales)}
\NormalTok{p }\SpecialCharTok{+} \FunctionTok{geom\_point}\NormalTok{() }\SpecialCharTok{+} 
  \FunctionTok{geom\_smooth}\NormalTok{(}\AttributeTok{method =} \StringTok{"lm"}\NormalTok{) }\SpecialCharTok{+} 
  \FunctionTok{scale\_x\_log10}\NormalTok{(}\AttributeTok{labels =}\NormalTok{ scales}\SpecialCharTok{::}\NormalTok{dollar)}
\NormalTok{p }\SpecialCharTok{+} \FunctionTok{geom\_point}\NormalTok{() }\SpecialCharTok{+} 
  \FunctionTok{geom\_smooth}\NormalTok{(}\AttributeTok{method =} \StringTok{"lm"}\NormalTok{) }\SpecialCharTok{+} 
  \FunctionTok{scale\_x\_log10}\NormalTok{(}\AttributeTok{labels =}\NormalTok{ scales}\SpecialCharTok{::}\NormalTok{)}
\end{Highlighting}
\end{Shaded}

\textbf{Exercise:} What does the \texttt{dollar()} call do?

\begin{Shaded}
\begin{Highlighting}[]
\NormalTok{?}\FunctionTok{dollar}\NormalTok{()}
\end{Highlighting}
\end{Shaded}

\begin{verbatim}
## starting httpd help server ... done
\end{verbatim}

\textbf{Exercise:} How can you find other ways of relabeling the scales
when using \texttt{scale\_x\_log10()}?

\hypertarget{the-ggplot-recipe}{%
\subsubsection{The ggplot Recipe}\label{the-ggplot-recipe}}

\begin{enumerate}
\def\labelenumi{\arabic{enumi}.}
\tightlist
\item
  Tell the \texttt{ggplot()} function what our data is.
\item
  Tell \texttt{ggplot()} \emph{what} variables we want to map to
  features of plots. For convenience we will put the results of the
  first two steps in an object called \texttt{p}.
\item
  Tell \texttt{ggplot} \emph{how} to display relationships in our data.
\item
  Layer on geoms as needed, by adding them on the \texttt{p} object one
  at a time.
\item
  Use some additional functions to adjust scales, labels, tickmarks,
  titles.
\end{enumerate}

\begin{itemize}
\tightlist
\item
  The \texttt{scale\_}, \texttt{labs()}, and \texttt{guides()} functions
\end{itemize}

\hypertarget{mapping-aesthetics-vs-setting-them}{%
\subsubsection{Mapping Aesthetics vs Setting
them}\label{mapping-aesthetics-vs-setting-them}}

\begin{Shaded}
\begin{Highlighting}[]
\NormalTok{p }\OtherTok{\textless{}{-}} \FunctionTok{ggplot}\NormalTok{(}\AttributeTok{data =}\NormalTok{ gapminder,}
            \AttributeTok{mapping =} \FunctionTok{aes}\NormalTok{(}\AttributeTok{x =}\NormalTok{ gdpPercap, }\AttributeTok{y =}\NormalTok{ lifeExp, }\AttributeTok{color =} \StringTok{\textquotesingle{}red\textquotesingle{}}\NormalTok{))}

\NormalTok{p }\SpecialCharTok{+} \FunctionTok{geom\_point}\NormalTok{() }\SpecialCharTok{+} \FunctionTok{scale\_x\_log10}\NormalTok{()}
\end{Highlighting}
\end{Shaded}

\includegraphics{lecture2-for-posting_files/figure-latex/mapping_vs_setting-1.pdf}

This is interesting (or annoying): the points are not yellow. How can we
tell ggplot to draw yellow points?

\begin{Shaded}
\begin{Highlighting}[]
\NormalTok{p }\OtherTok{\textless{}{-}} \FunctionTok{ggplot}\NormalTok{(}\AttributeTok{data =}\NormalTok{ gapminder,}
            \AttributeTok{mapping =} \FunctionTok{aes}\NormalTok{(}\AttributeTok{x =}\NormalTok{ gdpPercap, }\AttributeTok{y =}\NormalTok{ lifeExp, }\AttributeTok{color =}\NormalTok{ year))}
\NormalTok{p }\SpecialCharTok{+} \FunctionTok{geom\_point}\NormalTok{() }\SpecialCharTok{+} \FunctionTok{scale\_x\_log10}\NormalTok{()}
\end{Highlighting}
\end{Shaded}

\textbf{Exercise:} Based on the discussion in Chapter 3 of \emph{Data
Visualization} (read it), describe in your words what is going on. One
way to avoid such mistakes is to read arguments inside
\texttt{aes(\textless{}property\textgreater{}\ =\ \textless{}variable\textgreater{})}as
\emph{the property in the graph is determined by the data in }.

\textbf{Exercise:} Write the above sentence for the original call
\texttt{aes(x\ =\ gdpPercap,\ y\ =\ lifeExp,\ color\ =\ \textquotesingle{}yellow\textquotesingle{})}.

Aesthetics convey information about a variable in the dataset, whereas
setting the color of all points to yellow conveys no information about
the dataset - it changes the appearance of the plot in a way that is
independent of the underlying data.

Remember: \texttt{color\ =\ \textquotesingle{}yellow\textquotesingle{}}
and \texttt{aes(color\ =\ \textquotesingle{}yellow\textquotesingle{})}
are very different, and the second makes usually no sense, as
\texttt{\textquotesingle{}yellow\textquotesingle{}} is treated as
\emph{data} that has the same value always -- namely the value `yellow'.

\begin{Shaded}
\begin{Highlighting}[]
\NormalTok{p }\OtherTok{\textless{}{-}} \FunctionTok{ggplot}\NormalTok{(}\AttributeTok{data =}\NormalTok{ gapminder,}
            \AttributeTok{mapping =} \FunctionTok{aes}\NormalTok{(}\AttributeTok{x =}\NormalTok{ gdpPercap, }\AttributeTok{y =}\NormalTok{ lifeExp))}
\NormalTok{p }\SpecialCharTok{+} \FunctionTok{geom\_point}\NormalTok{() }\SpecialCharTok{+} \FunctionTok{geom\_smooth}\NormalTok{(}\AttributeTok{color =} \StringTok{"orange"}\NormalTok{, }\AttributeTok{se =} \ConstantTok{FALSE}\NormalTok{, }\AttributeTok{size =} \DecValTok{8}\NormalTok{, }\AttributeTok{method =} \StringTok{"lm"}\NormalTok{) }\SpecialCharTok{+} \FunctionTok{scale\_x\_log10}\NormalTok{()}
\end{Highlighting}
\end{Shaded}

\begin{verbatim}
## `geom_smooth()` using formula 'y ~ x'
\end{verbatim}

\includegraphics{lecture2-for-posting_files/figure-latex/exercise_args_for_smooth-1.pdf}

\textbf{Exercise:} Write down what all those arguments in
\texttt{geom\_smooth(...)} do.

\begin{Shaded}
\begin{Highlighting}[]
\NormalTok{p }\SpecialCharTok{+} \FunctionTok{geom\_point}\NormalTok{(}\AttributeTok{alpha =} \FloatTok{0.3}\NormalTok{) }\SpecialCharTok{+} 
  \FunctionTok{geom\_smooth}\NormalTok{(}\AttributeTok{method =} \StringTok{"gam"}\NormalTok{) }\SpecialCharTok{+} 
  \FunctionTok{scale\_x\_log10}\NormalTok{(}\AttributeTok{labels =}\NormalTok{ scales}\SpecialCharTok{::}\NormalTok{dollar) }\SpecialCharTok{+}
  \FunctionTok{labs}\NormalTok{(}\AttributeTok{x =} \StringTok{"GDP Per Capita"}\NormalTok{, }\AttributeTok{y =} \StringTok{"Life Expectancy in Years"}\NormalTok{,}
       \AttributeTok{title =} \StringTok{"Economic Growth and Life Expectancy"}\NormalTok{,}
       \AttributeTok{subtitle =} \StringTok{"Data Points are country{-}years"}\NormalTok{,}
       \AttributeTok{caption =} \StringTok{"Source: Gapminder"}\NormalTok{)}
\end{Highlighting}
\end{Shaded}

\begin{verbatim}
## `geom_smooth()` using formula 'y ~ s(x, bs = "cs")'
\end{verbatim}

\includegraphics{lecture2-for-posting_files/figure-latex/gapminder_with_labels-1.pdf}

Coloring by continent:

\begin{Shaded}
\begin{Highlighting}[]
\FunctionTok{library}\NormalTok{(scales)}
\end{Highlighting}
\end{Shaded}

\begin{verbatim}
## 
## Attaching package: 'scales'
\end{verbatim}

\begin{verbatim}
## The following object is masked from 'package:purrr':
## 
##     discard
\end{verbatim}

\begin{verbatim}
## The following object is masked from 'package:readr':
## 
##     col_factor
\end{verbatim}

\begin{Shaded}
\begin{Highlighting}[]
\NormalTok{p }\OtherTok{\textless{}{-}} \FunctionTok{ggplot}\NormalTok{(}\AttributeTok{data =}\NormalTok{ gapminder,}
            \AttributeTok{mapping =} \FunctionTok{aes}\NormalTok{(}\AttributeTok{x =}\NormalTok{ gdpPercap, }\AttributeTok{y =}\NormalTok{ lifeExp, }\AttributeTok{color =}\NormalTok{ continent, }\AttributeTok{fill =}\NormalTok{ continent))}
\NormalTok{p }\SpecialCharTok{+} \FunctionTok{geom\_point}\NormalTok{()}
\end{Highlighting}
\end{Shaded}

\includegraphics{lecture2-for-posting_files/figure-latex/gapminder_color_by_continent-1.pdf}

\begin{Shaded}
\begin{Highlighting}[]
\NormalTok{p }\SpecialCharTok{+} \FunctionTok{geom\_point}\NormalTok{() }\SpecialCharTok{+} \FunctionTok{scale\_x\_log10}\NormalTok{(}\AttributeTok{labels =}\NormalTok{ dollar)}
\end{Highlighting}
\end{Shaded}

\includegraphics{lecture2-for-posting_files/figure-latex/gapminder_color_by_continent-2.pdf}

\begin{Shaded}
\begin{Highlighting}[]
\NormalTok{p }\SpecialCharTok{+} \FunctionTok{geom\_point}\NormalTok{() }\SpecialCharTok{+} \FunctionTok{scale\_x\_log10}\NormalTok{(}\AttributeTok{labels =}\NormalTok{ dollar) }\SpecialCharTok{+} \FunctionTok{geom\_smooth}\NormalTok{()}
\end{Highlighting}
\end{Shaded}

\begin{verbatim}
## `geom_smooth()` using method = 'loess' and formula 'y ~ x'
\end{verbatim}

\includegraphics{lecture2-for-posting_files/figure-latex/gapminder_color_by_continent-3.pdf}

\textbf{Exercise:} What does \texttt{fill\ =\ continent} do? What do you
think about the match of colors between lines and error bands?

\begin{Shaded}
\begin{Highlighting}[]
\NormalTok{p }\OtherTok{\textless{}{-}} \FunctionTok{ggplot}\NormalTok{(}\AttributeTok{data =}\NormalTok{ gapminder,}
            \AttributeTok{mapping =} \FunctionTok{aes}\NormalTok{(}\AttributeTok{x =}\NormalTok{ gdpPercap, }\AttributeTok{y =}\NormalTok{ lifeExp))}
\NormalTok{p }\SpecialCharTok{+} \FunctionTok{geom\_point}\NormalTok{(}\AttributeTok{mapping =} \FunctionTok{aes}\NormalTok{(}\AttributeTok{color =}\NormalTok{ continent)) }\SpecialCharTok{+} \FunctionTok{geom\_smooth}\NormalTok{() }\SpecialCharTok{+} \FunctionTok{scale\_x\_log10}\NormalTok{()}
\end{Highlighting}
\end{Shaded}

\begin{verbatim}
## `geom_smooth()` using method = 'gam' and formula 'y ~ s(x, bs = "cs")'
\end{verbatim}

\includegraphics{lecture2-for-posting_files/figure-latex/gapminder_color_by_continent_single_line-1.pdf}

\textbf{Exercise:} Notice how the above code leads to a single smooth
line, not one per continent. Why?

\textbf{Exercise:} What is bad about the following example, assuming the
graph is the one we want? This is why you should set aesthetics at the
top level rather than at the individual geometry level if that's your
intent.

\begin{Shaded}
\begin{Highlighting}[]
\NormalTok{p }\OtherTok{\textless{}{-}} \FunctionTok{ggplot}\NormalTok{(}\AttributeTok{data =}\NormalTok{ gapminder,}
            \AttributeTok{mapping =} \FunctionTok{aes}\NormalTok{(}\AttributeTok{x =}\NormalTok{ gdpPercap, }\AttributeTok{y =}\NormalTok{ lifeExp))}
\NormalTok{p }\SpecialCharTok{+} \FunctionTok{geom\_point}\NormalTok{(}\AttributeTok{mapping =} \FunctionTok{aes}\NormalTok{(}\AttributeTok{color =}\NormalTok{ continent)) }\SpecialCharTok{+} 
  \FunctionTok{geom\_smooth}\NormalTok{(}\AttributeTok{mapping =} \FunctionTok{aes}\NormalTok{(}\AttributeTok{color =}\NormalTok{ continent, }\AttributeTok{fill =}\NormalTok{ continent)) }\SpecialCharTok{+} 
  \FunctionTok{scale\_x\_log10}\NormalTok{() }\SpecialCharTok{+} 
  \FunctionTok{geom\_smooth}\NormalTok{(}\AttributeTok{mapping =} \FunctionTok{aes}\NormalTok{(}\AttributeTok{color =}\NormalTok{ continent), }\AttributeTok{method =} \StringTok{"gam"}\NormalTok{)}
\end{Highlighting}
\end{Shaded}

\begin{verbatim}
## `geom_smooth()` using method = 'loess' and formula 'y ~ x'
\end{verbatim}

\begin{verbatim}
## `geom_smooth()` using formula 'y ~ s(x, bs = "cs")'
\end{verbatim}

\includegraphics{lecture2-for-posting_files/figure-latex/many_continents-1.pdf}

\hypertarget{additional-not-optional-exercises}{%
\subsection{Additional (not Optional)
Exercises}\label{additional-not-optional-exercises}}

\textbf{Exercise:} Find ways to save the figures that you made so that
you can use them elsewhere too. Create a new folder to save only images.
Use the command for saving to save the picture for the last image in
your new folder, after you have updated the axes, title, subtitle, and
caption of the image. Post your solution on Slack and use it to include
the final image above with a caption saying ``Saved by '' inside your
Slack message (see
\url{https://slack.com/help/articles/4403914924435-Add-descriptions-to-images}).
(Hint: \texttt{??save})

\textbf{Exercise:} Read section 3.8 ``Where to go next'' from DV. Based
on those ideas, experiment and create two different graphs with the
gapminder data. Describe each briefly in one sentence.

\textbf{Exercise:} Read section 1.6 of
\href{https://r4ds.had.co.nz/introduction.html}{R for Data Science} on
\emph{Getting help and learning more}. Go back to an error from your
previous assignment -- or pick a new one -- and post a reproducible
error as described in that section on the discourse forum.

\textbf{Exercise:} Do exercise 3.2.4 from
\href{https://r4ds.had.co.nz/data-visualisation.html\#first-steps}{R for
Data Science}. Include your code in chunks, describe the output and code
(where necessary) in the surrounding text.

\textbf{Exercise:} Go through Exercises in 3.3.1. If an exercise does
not make sense immediately (such as you don't know what a categorical
variable is), replace the question by a question that addresses that
point (in the case of the categorical variable ``What are categorical
and continuous variables and how are they different in R?''). Write it
down, try to answer that question, and ignore the original question.
That way you don't end up spending too much time on this one exercise.

\textbf{Exercise:} Read the (very short)
\href{https://r4ds.had.co.nz/workflow-basics.html}{Chapter 4 of R for
Data Science} and try exercise 1 in section 4.4.

\textbf{Bonus Exercise:} Why did I load the \texttt{scales} library
twice via \texttt{library(scales)} to knit?

\hypertarget{assignment-1}{%
\subsection{Assignment 1}\label{assignment-1}}

\begin{enumerate}
\def\labelenumi{\arabic{enumi}.}
\tightlist
\item
  Do the exercises in these and the previous lecture notes.
\item
  Knit lectures 2, making sure to get rid of those \texttt{eval=FALSE}
  that are just there because I didn't complete the code
\item
  Submit the pdf on the server
\end{enumerate}

\end{document}
